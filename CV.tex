%------------------------------------------------------------------------------
% CV in Latex
% Author : Charles Rambo
% Based off of: https://github.com/sb2nov/resume and Jake's Resume on Overleaf
% Most recently updated version may be found at https://github.com/fizixmastr 
% License : MIT
%------------------------------------------------------------------------------

\documentclass[A4,11pt]{article}
%\documentclass[letterpaper,11pt]{article} %For use in US
\usepackage{fontawesome5}
\usepackage{ragged2e}
\usepackage{latexsym}
\usepackage[empty]{fullpage}
\usepackage{titlesec}
\usepackage{marvosym}
\usepackage[usenames,dvipsnames,svgnames]{xcolor}
\usepackage{verbatim}
\usepackage{enumitem}
\usepackage[hidelinks]{hyperref}
\usepackage[english]{babel}
\usepackage{tabularx}
\usepackage{tikz}
\usetikzlibrary{shapes,positioning}
\input{glyphtounicode}


\begin{comment}
I am by no means a professional when it comes to the CV's/resumes, I have
received various trainings on how to write a CV and resume from my high 
school, as well as the Austin College and University of Eastern Finland's
career counseling departments. As I intend to share my CV as a template, I 
feel that it is my responsibility to provide explanations of my work.
\end{comment}


%-----FONT OPTIONS-------------------------------------------------------------
\begin{comment}
The font of the document will impact not just how readable it is, but how it is
perceived. In the "The Craft of Scientific Writing" by Michael Alley, shares a
common fonts for publication as well as their use. I have chosen to use
Palatino for its legibility, some others are given below. There is far too much
about typography to discus here. Note: serif fonts have short projecting
strokes, sans-serif fonts are sans (without) these strokes.
\end{comment}


% serif
 \usepackage{palatino}
% \usepackage{times} %This is the default as well
% \usepackage{charter}

% sans-serif
% \usepackage{helvet}
% \usepackage[sfdefault]{noto-sans}
% \usepackage[default]{sourcesanspro}

%-----PAGE SETUP---------------------------------------------------------------

% Adjust margins
\addtolength{\oddsidemargin}{-1cm}
\addtolength{\evensidemargin}{-1cm}
\addtolength{\textwidth}{2cm}
\addtolength{\topmargin}{-1cm}
\addtolength{\textheight}{2cm}

% Margins for US Letter size
%\addtolength{\oddsidemargin}{-0.5in}
%\addtolength{\evensidemargin}{-0.5in}
%\addtolength{\textwidth}{1in}
%\addtolength{\topmargin}{-.5in}
%\addtolength{\textheight}{1.0in}

\urlstyle{same}

\raggedbottom
\raggedright
\setlength{\tabcolsep}{0cm}

% Sections formatting
\titleformat{\section}{
  \vspace{-4pt}\scshape\raggedright\large
}{}{0em}{}[\color{black}\titlerule \vspace{-5pt}]

% Ensure that .pdf is machine readable/ATS parsable
\pdfgentounicode=1

% ATS Optimization: Add metadata for better parsing
\hypersetup{
    pdfauthor={Pasindu Uduwela},
    pdftitle={Pasindu Uduwela - Software Engineer CV},
    pdfsubject={Software Engineer Resume},
    pdfkeywords={Software Engineer, Full Stack Developer, Computer Engineering, Python, Java, C\#, React, Angular, .NET Core, FastAPI, Docker, Kubernetes, AWS, Google Cloud}
}

% Color definitions
\definecolor{linkcolour}{RGB}{0,102,204}

%-----CUSTOM COMMANDS FOR FORMATTING SECTIONS----------------------------------
\newcommand{\CVItem}[1]{
  \item\small{
    {#1 \vspace{-2pt}}
  }
}

\newcommand{\CVSubheading}[4]{
  \vspace{-2pt}\item
    \begin{tabular*}{0.97\textwidth}[t]{l@{\extracolsep{\fill}}r}
      \textbf{#1} & #2 \\
      \small#3 & \small #4 \\
    \end{tabular*}\vspace{-7pt}
}

\newcommand{\CVSubSubheading}[2]{
    \item
    \begin{tabular*}{0.97\textwidth}{l@{\extracolsep{\fill}}r}
      \text{\small#1} & \text{\small #2} \\
    \end{tabular*}\vspace{-7pt}
}

\newcommand{\CVSubItem}[1]{\CVItem{#1}\vspace{-4pt}}

\renewcommand\labelitemii{$\vcenter{\hbox{\tiny$\bullet$}}$}

\newcommand{\CVSubHeadingListStart}{\begin{itemize}[leftmargin=0.5cm, label={}]}
% \newcommand{\resumeSubHeadingListStart}{\begin{itemize}[leftmargin=0.15in, label={}]} % Uncomment for US
\newcommand{\CVSubHeadingListEnd}{\end{itemize}}
\newcommand{\CVItemListStart}{\begin{itemize}}
\newcommand{\CVItemListEnd}{\end{itemize}\vspace{-5pt}}
\newcommand{\CVSubheadingDetail}[1]{
  \item[]\small{\textit{#1}}
  \vspace{2pt}
}

% Technology card command - creates clickable rounded boxes (ATS-friendly)
% Usage: \techcard[optional-url]{Technology Name}
% Text content is ATS-readable as it's actual text, not images
% ATS Optimization: Add invisible text after card for better parsing
\newcommand{\techcard}[2][]{%
  \tikz[baseline=(text.base)]{%
    \node[draw=black!60, fill=black!5, rounded corners=4pt, inner sep=5pt, outer sep=10pt, anchor=base, font=\small] (text) {%
      \ifx\relax#1\relax%
        \href{https://www.google.com/search?q=#2}{\textcolor{black!90}{#2}}%
      \else%
        \href{#1}{\textcolor{black!90}{#2}}%
      \fi%
    };%
  }%
  \hspace{4pt}%
  % ATS: Add invisible text for better keyword extraction
  \textcolor{white}{\fontsize{1pt}{1pt}\selectfont #2}%
}

% Technology list wrapper for proper alignment and wrapping (ATS-friendly)
\newcommand{\techlist}[2]{%
  \begingroup
  \def\temp{#1}%
  \def\empty{}%
  \ifx\temp\empty%
    \endgroup
    \hspace{0.8cm}\begin{minipage}[t]{\dimexpr\textwidth-0.8cm-0.5cm\relax}%
      \raggedright%
      \setlength{\parskip}{22pt}%
      \setlength{\baselineskip}{1.5\baselineskip}%
      \addtolength{\baselineskip}{2pt}%
      #2%
    \end{minipage}%
  \else%
    \endgroup
    \hspace{0.8cm}\textbf{#1:} \vspace{0.3ex}\\
    \hspace{0.8cm}\begin{minipage}[t]{\dimexpr\textwidth-0.8cm-0.5cm\relax}%
      \raggedright%
      \setlength{\parskip}{22pt}%
      \setlength{\baselineskip}{1.5\baselineskip}%
      \addtolength{\baselineskip}{2pt}%
      #2%
    \end{minipage}%
  \fi%
}


%------------------------------------------------------------------------------
% CV STARTS HERE  %
%------------------------------------------------------------------------------
\begin{document}

%-----HEADING------------------------------------------------------------------
%-----HEADING------------------------------------------------------------------
\begin{comment}
In Europe it is common to include a picture of ones self in the CV. Select
which heading appropriate for the document you are creating.
\end{comment}

\begin{minipage}[c]{0.05\textwidth}
\-\
\end{minipage}
\begin{minipage}[c]{0.2\textwidth}
\begin{tikzpicture}
  \clip (0,0) circle (1.75cm);
  \node at (0,0) {\includegraphics[width=3.5cm]{pasinduuduwela}};
\end{tikzpicture}
\hfill\vline\hfill

\end{minipage}
\begin{minipage}[c]{0.6\textwidth}
    \textbf{\Huge \scshape{Pasindu Uduwela}} \\ \vspace{1pt} 
    % \scshape sets small capital letters, remove if desired
    \small{+94 75-77-66-896} \\
    \href{mailto:you@provider.com}{\underline{pasindudhanushka365@gmail.com}}\\
    % Be sure to use a professional *personal* email address
    \href{https://www.linkedin.com/in/pasindu-uduwela-24a34b159/}{\underline{linkedin.com/in/pasindu-uduwela-24a34b159}} \\
    % you should adjust you linked in profile name to be professional and recognizable
    \href{https://github.com/dhanushka365}{\underline{github.com/dhanushka365}}\\
    \href{https://dhanushka365.github.io/portfolio/}{\underline{dhanushka365.github.io/portfolio/}}
\end{minipage}

% Without picture
%\begin{center}
%    \textbf{\Huge \scshape Charles Rambo} \\ \vspace{1pt} %\scshape sets small capital letters, remove if desired
%    \small +1 123-456-7890 $|$ 
%    \href{mailto:you@provider.com}{\underline{you@provider.com}} $|$\\
%    % Be sure to use a professional *personal* email address
%    \href{https://linkedin.com/in/your-name-here}{\underline{linkedin.com/in/charles-rambo}} $|$
%    % you should adjust you linked in profile name to be professional and recognizable
%    \href{https://github.com/fizixmastr}{\underline{github.com/fizixmastr}}
%\end{center}




\begin{comment}
This CV was written for specifically for positions I was applying for in
academia, and then modified to be a template.

A standard CV is about two pages long where as a resume in the US is one page.
sections can be added and removed here with this in mind. In my experience, 
education, and applicable work experience and skills are the most import things
to include on a resume. For a CV the Europass CV suggests the categories: Work
Experience, Education and Training, Language Skills, Digital Skills,
Communication and Interpersonal Skills, Conferences and Seminars, Creative Works
Driver's License, Hobbies and Interests, Honors and Awards, Management and
Leadership Skills, Networks and Memberships, Organizational Skills, Projects,
Publications, Recommendations, Social and Political Activities, Volunteering.

Your goal is to convey a who, what , when, where, why for every item you share. 
The who is obviously you, but I believe the rest should be done in that order.
For example below. An employer cares most about the degree held and typically 
less about the institution or where it is located (This is still good 
information though). Whatever order you choose be consistent throughout.
\end{comment}
%-----PROFILE------------------------------------------------------------------
%-----PROFILE------------------------------------------------------------------
\section{Profile}
\begin{itemize}[leftmargin=0.5cm, label={}]
  \small{\item{
    Experienced Full Stack Software Engineer and Computer Engineer with over three years of industry experience. Specialized in Full Stack Development, System Design, AI/ML integration, and Cloud Computing. Proficient in Python, Java, C\#, JavaScript, React, Angular, .NET Core, FastAPI, Docker, Kubernetes, AWS, and Google Cloud Platform. Strong expertise in software architecture, database management, microservices, RESTful APIs, and DevOps practices. Proven track record in developing scalable web applications, implementing CI/CD pipelines, and delivering high-quality software solutions.
  }}
\end{itemize}



%-----EDUCATION----------------------------------------------------------------
%-----EDUCATION----------------------------------------------------------------
\section{Education}
 \vspace{0.8ex}
  \CVSubHeadingListStart
%    \CVSubheading % Example
%      {Degree Achieved}{Years of Study}
%      {Institution of Study}{Where it is located}
 \CVSubheading
      {{BSc. Eng. (Hons) in Computer Engineering }}{2019 -- 2023}
      {Faculty of Engineering University of Sri Jayewardenepura}{ Mount Lavinia, Sri Lanka}
      \CVSubheadingDetail{\\
         Class : \textit{Second Class Honors (Upper Division)} \\
         Minor : \textit{Data Management.} \\
         GPA(Overall- 8 semesters) : \textit{3.37/4.0.}\\
      \textbf{\scriptsize
        \faEnvelope\ \href{https://drive.google.com/file/d/190pLL1bQAe9CSid24-gzhgz9aYEGhJVG/view?usp=sharing}{\textcolor{gray}{Degree Certificate}}% 
        \hspace{1em}%
        \faEnvelope\ \href{https://drive.google.com/file/d/119aYmzniBQKrmUgvoSJvoVdMBIaxOeel/view?usp=sharing}{\textcolor{gray}{Academic Transcript}}
    }\\
    {\color{gray!40}\dotfill} % <-- very light dotted line
}

      
 \CVSubheading
      {{Diploma in Software Engineering (D.S.E-17.2)}}{2017 -- 2018}
      {National Institute of Business Management} {Colombo 07, Sri Lanka}
      \CVSubheadingDetail{\\
      GPA : \textit{4.0/4.0}\\
      \textbf{\scriptsize
            \faEnvelope\ \href{https://drive.google.com/file/d/1bTlTByw0UDZUXWVTEHyhxjHYYsY7VLuA/view?usp=sharing}{\textcolor{gray}{Degree Certificate}}%
            \hspace{1.5em}%
            \faEnvelope\ \href{https://drive.google.com/file/d/1lnRJzF5FUeuenHGWh97bLEvnG48V02as/view?usp=sharing}{\textcolor{gray}{Academic Transcript}}
        }\\
        {\color{gray!40}\dotfill} % <-- very light dotted line
      }
      
 \CVSubheading
      {G.C.E Advanced Level Examination (2017)}{2013 -- 2015}
      {Taxila Central College} {Horana, Sri Lanka}
      \CVSubheadingDetail{\\
       Result: \textit{Combined Math-A, Physics-B, Chemistry-  B} \\
       Z-Score: \textit{1.7112} \\
       District Rank: \textit{59 (Kaluthara)}\\
       \textbf{\scriptsize
            \faEnvelope\ \href{https://drive.google.com/file/d/1j82atMErRTbh5_62FSAAySZJVA6_94rJ/view?usp=sharing}{\textcolor{gray}{A/L Certificate}}%
            \hspace{1.5em}%
            \faEnvelope\ \href{https://drive.google.com/file/d/1fLuPGvjfLRqfevEc3SW4sXBAeBzgNXz-/view?usp=sharing}{\textcolor{gray}{GIT Certificate}}
        }\\
        {\color{gray!40}\dotfill} % <-- very light dotted line
       }
       
 \CVSubheading
      {G.C.E. O/L Examination (2012))}{2001 -- 2012}
      {Don Pedrick College (Horana)} {Horana, Sri Lanka}
      \CVSubheadingDetail{\\
      Result: 7 Distinctions (7 A's) and 2 Very Good passes (2 B's), including ICT\\
       \textbf{\scriptsize
            \faEnvelope\ \href{https://drive.google.com/file/d/1j82atMErRTbh5_62FSAAySZJVA6_94rJ/view?usp=sharing}{\textcolor{gray}{O/L Certificate}}
        }
      }
  \CVSubHeadingListEnd

%-----WORK EXPERIENCE----------------------------------------------------------
%-----WORK EXPERIENCE----------------------------------------------------------
\begin{comment}
try to briefly explain what you did and why it is relevant to the position you
are seeking
\end{comment}

\section{Work Experience}
 \vspace{0.8ex}
  \CVSubHeadingListStart
    \CVSubheading
      {Software Engineer}{Jul 2025 -- Feb 2026}
      { Huex GmbH}{Germany, Remote}
       \vspace{0.5ex}
      \CVItemListStart
        \CVItem{Worked as a System Designer, managing and maintaining Google Cloud VM instances, ensuring high availability, scalability, and security.}
        \CVItem{Designed and implemented an AI-driven interview platform using advanced context engineering principles in the HR \& Talent Intelligence domain.}
        \CVItem{Applied an agentic AI approach to build an automated, self-improving interview question bank system.}
        \CVItem{Integrated AI-powered question banks directly into live interviews to dynamically generate role- and skill-specific interview flows.}
        \CVItem{Integrated Google Gemini models to conduct AI-powered interviews with deep contextual understanding.}
        \CVItem{Designed and implemented intelligent question-and-answer evaluation mechanisms to accurately assess candidate responses.}
        \CVItem{Designed and developed interview scoring, analytics, and candidate ranking systems to enable data-driven hiring decisions.}
        \CVItem{Designed and developed backend services using FastAPI, following clean coding standards and a well-structured Repository Pattern architecture.}
        \CVItem{Developed frontend components using React by translating Figma designs in to reusable, scalable UI components.}
        \CVItem{Implemented Clean Architecture principles in the frontend to ensure separation of concerns, maintainability, and testability.}
        \CVItem{Designed and integrated Retrieval-Augmented Generation (RAG) pipelines and agentic AI workflows to enhance platform intelligence.}
        \CVItem{Member of Emergency Response Team}\\
      \CVItemListEnd
       {\color{gray!40}\dotfill} % <-- very light dotted line
    \CVSubheading
      {Software Engineer}{January 2016 -- July 2016}
      {Drifting Desk LLC}{USA, TX}
    \vspace{0.5ex}
      \CVItemListStart
        \CVItem{Worked as a .NET Core backend developer on an AI-driven multi-tenant hotel management platform.}
        \CVItem{Developed booking optimization and competitor intelligence features for a European client.}
        \CVItem{Integrated cutting-edge AI, real-time data analysis, and automation to help hotel chains make smarter decisions.}
        \CVItem{Developed scalable CRUD modules with Kafka integration for seamless internal microservice communication.}
        \CVItem{Implemented Request Response and Fire-and-forget patterns for external system communication.}
        \CVItem{Built a centralized Anti-XSS middleware to enforce security across all frontend and backend APIs.}
        \CVItem{Developed a unified access management mechanism for consistent security enforcement.}
        \CVItem{Designed background jobs (Hangfire-style) for recurring tasks.}
        \CVItem{Implemented background jobs for large-scale data imports.}
        \CVItem{Implemented .NET Server-Sent Events (SSE) for real-time chart updates.}
        \CVItem{Implemented SSE for live table updates.}
        \CVItem{Wrote K6 performance tests to validate system reliability.}
        \CVItem{Validated system scalability through comprehensive performance testing.}
        \CVItem{Integrated Keycloak REST APIs for secure, multi-tenant authentication.}
        \CVItem{Implemented user management using Keycloak integration.}
        \CVItem{Researched and implemented Kafka consumer parallelism.}
        \CVItem{Enhanced real-time data processing capabilities through parallel consumer architecture.}
        \CVItem{Developed custom pagination logic for efficient rendering of large datasets.}
        \CVItem{Configured Bitbucket Pipelines to dynamically read Helm chart values from Bitbucket repository variables.}
        \CVItem{Enabled environment-specific deployments through automated pipeline configuration.}
        \CVItem{Integrated security vulnerability scans into the pipeline for early risk detection.}
        \CVItem{Improved CI/CD security compliance through automated vulnerability scanning.}
        \CVItem{Mentored junior developers on architecture and coding standards.}
        \CVItem{Provided guidance on secure development practices.}\\
      \CVItemListEnd
     {\color{gray!40}\dotfill} % <-- very light dotted line
    \CVSubheading
      {Associate Software Engineer}{August 2023 -- July 2024}
      {ICP Technologies (Pvt.) Ltd}{Sri Jayewardenepura Kotte, Sri Lanka}\\
     \vspace{2.8ex}
      \textbf{Advanced Accounting Software - Enterprise Management Domain (4 Months)}
      \begin{itemize}[leftmargin=0.8cm]
        \item Designed and implemented the backend core architecture.
        \item Implemented centralized audit and request logging using Spring AOP, Filters, and Interceptors.
        \item Developed RESTful APIs with comprehensive unit tests.
        \item Introduced dynamic queries for flexible data access.
        \item Mentored junior developers on best practices and architecture.
      \end{itemize}
      \techlist{Technologies \& Design Patterns}{%
        \techcard{Modular Architecture} \techcard{Spring Boot} \techcard{JPA Specification} \techcard{MySQL} \techcard{Docker} \techcard{Keycloak} \techcard{JWT} \techcard{JUnit} \techcard{JaCoCo} \techcard{Azure DevOps}%
      } \\
      \vspace{1.2ex}
      \textbf{USA Client Project - FinTech Domain (7+ Months)}
      \begin{itemize}[leftmargin=0.8cm]
        \item Worked as a Backend Developer at Jack Henry.
        \item Designed and implemented a Transactional Outbox pattern to ensure message reliability.
        \item Automated retention workers for efficient data management.
        \item Implemented retry mechanisms for failed transactions to improve system reliability.
        \item Resolved production issues and improved overall system stability.
      \end{itemize}
      \techlist{Technologies \& Design Patterns}{%
        \techcard{CQRS} \techcard{DDD} \techcard{Clean Architecture} \techcard{C\#} \techcard{.NET Core} \techcard{RabbitMQ} \techcard{MassTransit} \techcard{MS SQL} \techcard{Dapper} \techcard{MediatR} \techcard{JMeter} \techcard{Datadog} \techcard{Azure DevOps}%
      }\\
      \vspace{1.8ex}
      {\color{gray!40}\dotfill} % <-- very light dotted line
    \CVSubheading
      {Intern Software Engineer}{February 2022 -- August 2022}
      {Mitra Innovation (Pvt) Ltd}{Moratuwa, Sri Lanka}\\
      \vspace{1.8ex}
      \textbf{Triton by Dygisec - Cloud Security Domain (4 Months)}
      \begin{itemize}[leftmargin=0.8cm]
        \item Worked as a Golang backend developer, developing APIs and building ETL pipelines using AWS Lambda to PostgreSQL RDS.
        \item Created API endpoints for compliance dashboard service to retrieve hard-coded data from the backend.
        \item Documented all REST API endpoints using Swagger UI across all services in the project.
        \item Researched and built OpenID Connect \& OAuth2.0 plugin for Kong Gateway.
        \item Collaborated on filling compliance data into Google Sheets with 11 tables, investigating Triton Confluence pages.
        \item Worked with compliance repositories: AWS CIS v1.2.0, AWS Best Practice FW, and PCI DSS 3.2.1.
        \item Built a Golang CSV file reader program to automate manual data.go file creation from policy compliance Excel sheets.
        \item Modified Entity Relationship Diagram (ERD) for Compliance and rules repository according to principal architect's feedback.
        \item Modified data seeder after database structure changes and verified data seeding using K9s logs and DBeaver.
        \item Created Google Sheet documenting 250+ AWS managed config rules and shared with developers and architects.
        \item Designed and implemented ETL pipeline to read files from S3 bucket and load into AWS RDS PostgreSQL database.
        \item Automated ETL deployment process using AWS CloudFormation service with YAML configuration.
      \end{itemize}
      \techlist{Technologies \& Design Patterns}{%
        \techcard{Microservices} \techcard{Go} \techcard{Kong Gateway} \techcard{Helm} \techcard{Swagger} \techcard{Kubernetes} \techcard{Docker} \techcard{GORM} \techcard{AWS}%
      } \\
      \vspace{1.2ex}
      \textbf{ATS - HR \& Recruitment Domain (2 Months)}
      \begin{itemize}[leftmargin=0.8cm]
        \item Created relational database and designed ER diagram for ATS by observing entities through PeopleHR sandbox environment.
        \item Developed scheduler to sync PeopleHR data with the proposed ATS application.
        \item Designed wireframes for ATS using Figma UX/UI design tool.
        \item Collaborated with all developers on applicant tracking system project tasks.
        \item Designed project timeline and Gantt chart using Jira project management tools.
      \end{itemize}
      \techlist{Technologies}{%
        \techcard{Lucidchart} \techcard{Figma} \techcard{Angular} \techcard{PeopleHR APIs}%
      }
  \CVSubHeadingListEnd




%-----PROJECTS AND RESEARCH----------------------------------------------------
%-----PROJECTS AND RESEARCH----------------------------------------------------
\section{Projects and Research}
 \vspace{1.8ex}
  \CVSubHeadingListStart
    \CVSubheading
      {RevitDataExtractor}{2025}
      {Freelance Project (Hong Kong) \textbullet\ \faGithub\ \href{https://github.com/dhanushka365/RevitDataExtractor}{\textcolor{gray}{GitHub}}}{}
      \begin{itemize}[leftmargin=0.8cm]
        \item Developed an Autodesk Revit 2024 add-in (MVP) using C\#/.NET to automate building sustainability data extraction for BEAM Plus certification with a WPF UI and core extensible services.
      \end{itemize}
      \techlist{Technologies}{%
        \techcard{C\#} \techcard{.NET Framework 4.8} \techcard{WPF} \techcard{MVVM} \techcard{Google Maps API} \techcard{CSV Processing} \techcard{Authentication} \techcard{Auto-save} \techcard{Real-time Sensor Validation}%
      }\\
       \vspace{1.8ex}
      {\color{gray!40}\dotfill} % <-- very light dotted line
    \CVSubheading
      {Property Maintenance Management System (PMMS)}{2024}
      {Full Stack Web Application \textbullet\ \faGithub\ \href{https://github.com/dhanushka365/PMMS}{\textcolor{gray}{GitHub}}}{}
      \begin{itemize}[leftmargin=0.8cm]
        \item Developed a comprehensive property maintenance management system with role-based access control, RESTful API, and real-time status tracking for maintenance requests.
      \end{itemize}
      \techlist{Technologies}{%
        \techcard{Angular} \techcard{.NET Core} \techcard{C\#} \techcard{Clean Architecture} \techcard{RESTful API} \techcard{Docker} \techcard{Swagger} \techcard{Role-based Access Control} \techcard{Image Upload} \techcard{Search \& Filter}%
      }\\
       \vspace{1.8ex}
      {\color{gray!40}\dotfill} % <-- very light dotted line
    
    \CVSubheading
      {WLAN Edge Computers}{2023}
      {Group Project \textbullet\ \faGithub\ \href{https://github.com/dhanushka365/wlan-app}{\textcolor{gray}{GitHub}}}{}
      \begin{itemize}[leftmargin=0.8cm]
        \item Edge-computing-based system integrating surveillance video processing with smart energy metering.
      \end{itemize}
      \techlist{Technologies}{%
        \techcard{Python} \techcard{Flask} \techcard{Arduino} \techcard{Machine Learning} \techcard{Deep Learning} \techcard{PostgreSQL} \techcard{Raspberry Pi} \techcard{Laravel}%
      }\\
       \vspace{1.8ex}
      {\color{gray!40}\dotfill} % <-- very light dotted line
    
    \CVSubheading
      {School Ease}{2023}
      {Group Project \textbullet\ \faGithub\ \href{https://github.com/dhanushka365/SchoolEase}{\textcolor{gray}{GitHub}}}{}
      \begin{itemize}[leftmargin=0.8cm]
        \item Microservice-based school management backend system.
      \end{itemize}
      \techlist{Technologies}{%
        \techcard{Spring Boot} \techcard{Eureka} \techcard{Zipkin} \techcard{Docker} \techcard{Kubernetes} \techcard{PostgreSQL} \techcard{JWT Authentication}%
      }\\
       \vspace{1.8ex}
      {\color{gray!40}\dotfill} % <-- very light dotted line
    
    \CVSubheading
      {Smart Meter for Home Water Management}{2021}
      {Group Project \textbullet\ \faGithub\ \href{https://github.com/dhanushka365/SMART-METER}{\textcolor{gray}{GitHub}}}{}
      \begin{itemize}[leftmargin=0.8cm]
        \item IoT-based real-time water usage monitoring system with web and desktop applications.
      \end{itemize}
      \techlist{Technologies}{%
        \techcard{Arduino} \techcard{MySQL} \techcard{HTML} \techcard{CSS} \techcard{JavaScript} \techcard{C\#} \techcard{Bunifu}%
      }
  \CVSubHeadingListEnd



%-----PERSONAL INFORMATION-----------------------------------------------------
%-----PERSONAL INFORMATION----------------------------------------------------
\section{Personal Information}
\begin{itemize}[leftmargin=0.5cm, label={}]
  \small{\item{
    Languages: \textbf{Sinhala} (Native proficiency), \textbf{English} (Professional Working proficiency)
  }}
\end{itemize}


%-----SKILLS-------------------------------------------------------------------
%-----SKILLS-------------------------------------------------------------------
%-----LANGUAGES & FRAMEWORKS----------------------------------------------------
\section{Languages \& Frameworks}
\techlist{}{%
  \techcard{Python} \techcard{FastAPI} \techcard{Flask} \techcard{SQL} \techcard{MySQL} \techcard{MongoDB} \techcard{C} \techcard{C++} \techcard{Java} \techcard{C\#} \techcard{HTML} \techcard{PHP} \techcard{CSS} \techcard{JavaScript} \techcard{.NET Core} \techcard{Node.js} \techcard{Next.js} \techcard{Laravel} \techcard{Spring Boot} \techcard{Angular} \techcard{React} \techcard{Golang}%
}
\vspace{0.5ex}
%-----TOOLS & TECHNOLOGIES-----------------------------------------------------
\section{Tools \& Technologies}
\techlist{}{%
  \techcard{Figma} \techcard{Lucidchart} \techcard{Docker} \techcard{Kubernetes} \techcard{Helm} \techcard{Azure DevOps} \techcard{GitHub Actions} \techcard{Bitbucket} \techcard{Git} \techcard{Jira} \techcard{Azure} \techcard{Trello} \techcard{MySQL} \techcard{PostgreSQL} \techcard{MS SQL} \techcard{MongoDB} \techcard{AWS} \techcard{Google Cloud} \techcard{Kong Gateway} \techcard{Swagger} \techcard{REST APIs} \techcard{Keycloak} \techcard{JWT} \techcard{Kafka} \techcard{RabbitMQ} \techcard{MassTransit} \techcard{JUnit} \techcard{JaCoCo} \techcard{xUnit} \techcard{K6} \techcard{JMeter} \techcard{Datadog} \techcard{Zipkin} \techcard{Eureka} \techcard{GORM} \techcard{Dapper} \techcard{JPA Specification} \techcard{MediatR} \techcard{MinIO} \techcard{Firebase} \techcard{Redis} \techcard{Supabase} \techcard{Google Maps API} \techcard{Google Gemini} \techcard{Google Calendar API} \techcard{PeopleHR APIs} \techcard{SMTP} \techcard{Jitsi} \techcard{Arduino} \techcard{Raspberry Pi} \techcard{Bunifu}%
}
\vspace{0.5ex}
%-----CONCEPTS------------------------------------------------------------------
\section{Concepts}
\techlist{}{%
  \techcard{Distributed Systems} \techcard{Object Oriented Programming} \techcard{Data Structures and Algorithms} \techcard{Database Management} \techcard{Cloud Computing} \techcard{Machine Learning} \techcard{Computer Vision \& Image Processing} \techcard{Parallel Programming} \techcard{Computer Networks \& Security}%
}
\vspace{0.5ex}
%-----SOFT SKILLS--------------------------------------------------------------
\section{Soft Skills}
\techlist{}{%
  \techcard{Communication} \techcard{Team player attitude} \techcard{Presentation Skills} \techcard{Time management} \techcard{Research \& Development} \techcard{Fast Learner} \techcard{Decision Making} \techcard{Leadership} \techcard{Project management}%
}
\vspace{0.5ex}


%-----ACHIEVEMENTS & CERTIFICATIONS---------------------------------------------
%-----ACHIEVEMENTS & CERTIFICATIONS---------------------------------------------
\section{Achievements \& Certifications}
\begin{itemize}[leftmargin=0.5cm, label={}]
  \small{\item
    \textbf{SLIOT 2022 Finalist – Project Eleccare} \hfill \textit{2022} \\
    Competition Finalist \textbullet\ \faGithub\ \href{https://github.com/dhanushka365/PZEM-004T}{\textcolor{gray}{GitHub Repository}}
  }
  
  \small{\item
    \textbf{Introduction to Amazon Web Services} \hfill \textit{2021} \\
    Coursera \textbullet\ \faCertificate\ \href{https://www.coursera.org/account/accomplishments/certificate/87EX3LAXGSUG}{\textcolor{gray}{View Certificate}}
  }
  
  \small{\item
    \textbf{Google Analytics for Beginners} \hfill \textit{2021} \\
    Google Analytics Academy \textbullet\ \faCertificate\ \href{https://analytics.google.com/analytics/academy/certificate/0UZgS1-WSgSrWlgTt0JrEg}{\textcolor{gray}{View Certificate}}
  }
  
  \small{\item
    \textbf{Circuito'21 – PCB Designing} \hfill \textit{2021} \\
    Workshop/Certification \textbullet\ \faCertificate\ \href{https://drive.google.com/file/d/1YryQ0XLgUS_0o02BFWk6Isnw15sqggzv/view}{\textcolor{gray}{View Certificate}}
  }
  
  \small{\item
    \textbf{Creating Multi-Task Models with Keras} \hfill \textit{2021} \\
    Coursera \textbullet\ \faCertificate\ \href{https://www.coursera.org/account/accomplishments/verify/8VSEFSKB3BUK}{\textcolor{gray}{View Certificate}}
  }
\end{itemize}


%-----REFEREES-----------------------------------------------------------------
%-----REFEREES------------------------------------------------------------------
\section{Referees}
\begin{itemize}[leftmargin=0.5cm, label={}]
  \small{\item{
    \begin{tabular*}{\textwidth}{@{\extracolsep{\fill}}p{0.48\textwidth}p{0.48\textwidth}}
      \textbf{\faUser\ Krishnamenan Rasenthiran} & \textbf{\faUser\ Ranushka Pasindu} \\
      \textbf{\faCog\ Software Engineer} & \textbf{\faCog\ Senior Software Engineer} \\
      \faPhone\ \texttt{+94 778864273} & \faPhone\ \texttt{+94 711523365} \\
      \faEnvelope\ \href{mailto:krishnamenan@huex.com}{krishnamenan@huex.com} & \faEnvelope\ \href{mailto:rdharmaranga@gmail.com}{rdharmaranga@gmail.com}
    \end{tabular*}
  }}
\end{itemize}

    
%------------------------------------------------------------------------------
\end{document}